\section{Conclusions}
This paper has presented a novel, fast approach to automatically construct training data for event extraction with little human
involvement, which in turn allows effective event extraction modeling. To generate training data, our approach first extracts, from
existing structured knowledge bases, which of the arguments best describe an event; then, it uses the key arguments to automatically infer
the occurrence of an event without explicit trigger identification. To perform event extraction, we develop a novel architecture based on
neural networks and post inference, which does not require explicit trigger information. We apply our approach to label Wikipedia articles
using knowledge extracted from Freebase. We show that and the quality of the automatically generated training data is comparable to those
that were manually labeled by human experts. We demonstrate that this large volume of high-quality training data, in combination of our
novel event extraction architecture, not only lead to a highly effective event extractor, but also enable multi-typed event detection for
the first time.




%In this paper, we propose a novel event extraction paradigm without expert-designed event templates by leveraging structured knowledge bases or tables to automatically acquire event schema and corresponding training data. We propose a BLSTM-CRF model with ILP-based post inference to extract multi-typed event mentions without explicit trigger annotations. Experimental results on both manual and automatic evaluations show that it is possible to learn to identify both typed events and typed roles with indirect supervision from Freebase or Wikipedia tables.
%In the future, we will first further investigate how to better characterize key arguments for a certain event type, and extend our work to update a structured knowledge base according to news events.
%
%%Furthermore, our model can extract information not covered by Freebase which indicates the possibility to extend this work to knowledge base population.
