\begin{abstract}

The task of event extraction  %which detects the occurrence of events with specific types and extract arguments (i.e. typed participants or
%attributes) that are associated with an event, is an important technique that
%which underpins many natural language processing applications,
has long been investigated in a supervised learning paradigm, % is an effective method for building event extraction systems.
%the effectiveness of event extractors 
which is bound by the number and the quality of  the training instances. Existing  training data must be
manually generated through a combination of expert domain knowledge and extensive human involvement. However, due to drastic efforts
required in annotating text, the resulted datasets are usually small, which severally affects the quality of the learned model, making it
hard to generalize.
%
Our work develops an automatic approach for generating training data for event extraction. Our approach allows us to scale up event
extraction training instances from thousands to hundreds of thousands, and it does this at a much lower cost than a manual approach. We
achieve this by employing distant supervision to automatically create event annotations from unlabelled text using existing structured
knowledge bases or tables. We then develop a neural network model with post inference to transfer the knowledge extracted from structured
knowledge bases to automatically annotate typed events with corresponding  arguments in text. We evaluate our approach by using the
knowledge extracted from Freebase to label texts from Wikipedia articles. Experimental results show that our approach can generate a
large number of high-quality training instances. We show that this large volume of training data not only leads to a better event
extractor, but also allows us to detect multiple typed events. % -- a feature that none of the existing event extractors can offer.


%Supervised learning is an effective method for building event extraction systems. The effectiveness of the learning is bound by the
%number and the quality of  the training instances. Existing event extractor systems are typically built upon expert-annotated datasets.
%However, due to drastic efforts involved in annotating text, these datasets are usually small and cover only a limited variety of event
%types. In this paper, we investigate ways to generate training datasets that require little expert involvement but can cover a richer set
%of event types. We achieve this by employing distant supervision to automatically create event annotations from unlabelled text using
%existing structured knowledge bases or tables. We then develop a novel neural network model with post inference, to detect multi-typed
%event mentions \FIXME{ZW: reviewers may not know what is a event mention} with corresponding arguments. Experiments on the datasets
%collected through Freebase and Wikipedia tables show that it is possible to learn to extract events of rich types without human-annotated
%training data. \FIXME{ZW: Will come back to the experimental results later.}

%Existing event extraction systems are typically investigated in a supervised learning paradigm, which heavily relies on the quality of
%expert-annotated datasets. Due to the drastic efforts involved in text annotation, the human-annotated datasets are typically small, only
%covering a limited variety of event types. This limits This limits the quality of learned event extractor, making it hard to generalize.


%In this paper, we address the problem of automatically building event extractors for rich event types with little expert involvement. We
%achieve this by employing distant supervision to automatically create event annotations from unlabelled text using structured knowledge
%bases or tables. We propose a novel neural network model with post inference, to detect multi-typed event mentions with corresponding
%arguments.
%% We evaluate our approach by investigating the feasibility of  automatically collecting training data for event extraction from both
%Experiments on the datasets collected through Freebase and Wikipedia tables show
%that
%%our proposed extraction model is designed to identify both typed event mentions and typed arguments.
%% Both automatic and manual evaluations demonstrate that
%it is possible to learn to extract events of rich types without human-annotated training data.

%
%and rely on expert-annotated datasets, with limited event types.
%%such as ACE and ERE event extraction frameworks.
%However, designing and constructing these
%high-quality corpora, usually with limited size and coverage of event types,  is costly, which
%makes learned extractors hard to generalize.  With the essence of distant supervision,
%%Inspired by some Freebase schemas which share similar structures with ACE event templates,
%we investigate the possibilities of automatic construction of training data for various event types
%with the help of structured knowledge bases.
%the following problems in this paper: can we generate a feasible dataset for event extraction with Freebase automatically and is it possible to extract events on this dataset.
%We first propose four hypotheses based on our observation and produce our dataset accordingly. Then,
%We further propose a novel neural network with ILP-based post inference, committing to
%handling two challenges in event extraction: multi-type events and multi-word arguments.
%Both automatic and manual evaluations demonstrate that it is possible to learn to extract various  events, according to existing knowledge bases, without human-annotated training data.
\end{abstract}
