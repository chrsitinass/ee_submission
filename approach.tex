\section{Training Data Generation}
Our approach exploits structural information like \FB \CVT tables to  automatically annotate event mentions\footnote{An event mention is a
phrase or sentence within which an event is described, including its type and arguments.} in order to generate training data to learn an
event extractor. Such an approach is known as Distant Supervision (\DS)\FIXME{~\cite{}}. Essentially, we use the knowledge extracted from a
known knowledge base to label sentences from another dataset to generate training data.


We use the arguments of a \CVT table entry to infer what event a sentence is likely to express. A \CVT table entry can have multiple
arguments but not all of the arguments are useful in annotation. For example, the \texttt{divisions\_formed} argument in
Figure~\ref{fig:example} (b) is not as important as the other three arguments when determining if a sentence expresses a
\texttt{business.acquisition} event. Therefore, our first step is to identify the key arguments from a \CVT table entry. A key argument is
an argument that plays an important role in one event, which helps to distinguish with other events. If a sentence contains all key
arguments of an entry in an event table (e.g. a \CVT table), it is likely to express the event expressed by the table entry. If a sentence
is labelled as an event mention of a \CVT event, we also record the words or phrases that match the entry’s properties as the involved
arguments, with the roles specified by their corresponding property names. For instance, sentence S1 shown in Figure~\ref{fig:example} (a)
is a mention of \texttt{business.acquisition} type, its involved arguments are ``Remedy Corp", ``BMC Software", and ``2004", and the roles
of the three arguments are \texttt{company\_acquired}, \texttt{acquiring\_company} and \texttt{date} respectively.

In addition to use the raw information given by a \CVT entry, we also use alias information (such as Wikipedia redirect) to match two
arguments that have different literal name but refer to the same entity (e.g. Microsoft and MS).

\subsection{Determining Key Arguments}
We use the following formula to calculate the importance value, $I_{cvt, arg}$, of an argument \emph{arg} (e.g., date) to its event type
\emph{cvt} (e.g., \texttt{business.acquisition}):

\begin{equation}
	I_{cvt, arg} = log \frac{count(cvt, arg)}{count(cvt) \times count(arg)}
\end{equation}


where $count(cvt)$ is the number of instances of type $cvt$ within a \CVT table, $count(arg)$ is the number of times $arg$ appearing in all
\CVT types within a \CVT table, and $count(cvt, arg)$ is the number of $cvt$ instances that contain $arg$ across all \CVT tables.
\FIXME{explain how do you come up with this formula.}


Our strategy for selecting key arguments of a given event type is described as follows.

\begin{description}

\item [P1] For a \CVT table with $n$ arguments, we first calculate the important value of each argument. Next, we sort the arguments
    according to their importance values in descending order, so that arguments with the higher importance values will appear on the top
    of the list. We then consider the top half $\ceil[\big]{n}$ (rounding up) arguments on the sorted list as key arguments.

\item [P2] We find that time-related arguments are useful in determining the event type, so we always include time-related arguments
    (such as date) in the key argument set.

\item [P3] We also remove sentences from the generated dataset in which the dependency distances between any two key arguments are
    greater than 2.

\end{description}

Using this strategy, the first three argument of the CVT entry are considered to be key arguments for event typ
\texttt{business.acquisition}.


\subsection{Key Argument Selection Parameters}
To determine how many arguments should be chosen (step P1 in our strategy), we have conducted a series of evaluations on the quantity and
quality of the datasets using different key argument selection policies. We found that the using the top half arguments sorted by their
important scores gives the best accuracy.

We use three example sentences from the Wiki text dataset to explain steps P2 and P3 in our key argument selection strategy described
above. The three sentences are:

\begin{quote}
\textbf{S2}: \underline{\emph{Microsoft}} spent \$6.3 billion buying online display advertising company \underline{\emph{aQuantive}} in
\underline{\emph{2007}}.
\end{quote}
\begin{quote}
\textbf{S3}: Microsoft hopes aQuantive's Brian McAndrews can outfox Google.
\end{quote}
\begin{quote}
\textbf{S4}: On April 29th, Elizabeth II and Prince Philip witnessed the marriage of Prince William.
\end{quote}

\begin{table}
 \scriptsize
 \caption{\CVT entry of \texttt{business.acquisition} in \FB. \label{tbl:bs}}
        \begin{tabular}{llllc}
        \toprule
        id & company\_acquired & acquiring\_company & date & divisions\_formed\\
        \midrule
        m.05nb3y7 & aQuantive & Microsoft & 2007 & N/A\\
        \bottomrule
        \end{tabular}
\end{table}


\begin{figure}
\centering
	\includegraphics[width=.48\textwidth]{figure2.png}
	\caption{The dependency tree of S3, which partially matches a \CVT entry of \emph{people.marriage} from \FB. \label{fig:2}}
\end{figure}


We regard a sentence as \emph{positive} when it mentions the occurrence of an event, or  \emph{negative} otherwise. For example, S1 in
Figure~\ref {fig:example} (a) and S2 (with its arguments in italics and underlined) are positive examples, while S3 and S4 are negative. We
wish to minimize the number of negative sentences to be labelled as positive (i.e., false positive) and to maximize  the number of true
positive sentences to be labelled as their correct events.

Although time-related arguments are often missing in the currently imperfect \KBs, they are crucial to identify the actual occurrence of an
event. As an example, suppose we want to use the \CVT entry shown in Table~\ref{tbl:bs} to determine if sentence S3 is a
\texttt{business.acquistion} event. This sentence contains \emph{Microsoft} as \texttt{acquiring\_company} and \emph{aQuantive} as
\texttt{company\_acquired}. If we ignore the time-related argument (i.e., date in this case), this setence could be mistakenly considered
as a positive sample for event \texttt{business.acquisition}. Therefore, we always consider time-related arguments as key arguments (step
P2) if these are given on the \CVT entry.

Finally, step P3 is based on our intuitions that two arguments involved in the same event mention are likely to be closer within the
syntactic structure. As an example, consider S4 and its dependency parse tree shown in Figure~\ref{fig:2}. Although both \emph{Prince
Philip} and \emph{marriage} can be matched as key arguments in a  \texttt{people.marriage} entry, but with a distance of 3 (i.e., far from
each other under our criterion) on the dependency parse tree, thus S4 will be labelled as negative.
