\section{Our Approach}
Our approach exploits structural information like \FB \CVT tables to  automatically annotate event mentions\footnote{An event mention is a
phrase or sentence within which an event is described, including its type and arguments.} in order to generate training data to learn an
event extractor. Such an approach is known as Distant Supervision (\DS)\FIXME{~\cite{}}. Essentially, we use the knowledge extracted from a
known knowledge base to label sentences from another dataset to generate training data.


We use the arguments of a \CVT table entry to infer what event a sentence is likely to express. A \CVT table entry can have multiple
arguments but not all of the arguments are useful in annotation. For example, the \texttt{divisions\_formed} argument in
Figure~\ref{fig:example} (b) is not as important as the other three arguments when determining if a sentence expresses a
\texttt{business.acquisition} event. Therefore, our first step is to identify the key arguments from a \CVT table entry. A key argument is
an argument that plays an important role in one event, which helps to distinguish with other events. If a sentence contains all key
arguments of an entry in an event table (e.g. a \CVT table), it is likely to express the event expressed by the table entry. If a sentence
is labelled as an event mention of a \CVT event, we also record the words or phrases that match the entry’s properties as the involved
arguments, with the roles specified by their corresponding property names. For instance, sentence S1 shown in Figure~\ref{fig:example} (a)
is a mention of \texttt{business.acquisition} type, its involved arguments are ``Remedy Corp", ``BMC Software", and ``2004", and the roles
of the three arguments are \texttt{company\_acquired}, \texttt{acquiring\_company} and \texttt{date} respectively.

\subsection{Determining Key Arguments}
We use the following formula to calculate the importance value, $I_{cvt, arg}$ of an argument \emph{arg} (e.g., date) to its event type
\emph{cvt} (e.g., \texttt{business.acquisition}):

\begin{equation}
	I_{cvt, arg} = log \frac{count(cvt, arg)}{count(cvt) \times count(arg)}
\end{equation}


where $count(cvt)$ is the number of instances of type $cvt$ in a \CVT table, $count(arg)$ is the number of times $arg$ appearing in all
\CVT types, and $count(cvt, arg)$ is the number of $cvt$ instances that contain $arg$.
